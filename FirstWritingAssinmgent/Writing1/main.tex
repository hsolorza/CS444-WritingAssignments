\documentclass[10pt, draftclsnofoot,onecolumn]{IEEEtran}

%% Language and font encodings

%% Sets page size and margins
\usepackage[a4paper,top=3cm,bottom=2cm,left=3cm,right=3cm,marginparwidth=0.75cm]{geometry}

%% Useful packages
\usepackage{amsmath}
\usepackage{cite}
\usepackage{listings}
\usepackage{graphicx}
\usepackage{url}
\usepackage[colorinlistoftodos]{todonotes}
\usepackage[colorlinks=true, allcolors=blue]{hyperref}

\lstset{frame=tb,
  language=C,
  aboveskip=3mm,
  belowskip=3mm,
  showstringspaces=false,
  columns=flexible,
  basicstyle={\small\ttfamily},
  numbers=none,
  keywordstyle=\color{blue},
  breaklines=true,
  breakatwhitespace=true,
  tabsize=3
}

\title{CS 444 Writing Assignment 1 \\ {\large Spring 2017}}
\author{Hannah Solorzano}

\begin{document}

\maketitle

\vfill
\begin{abstract}
There are considerable differences in how Windows and FreeBSD handle processes,
threads, and scheduling in comparison to Linux. This paper explains how they differ,
any similarities, and why these similarities exist.
\end{abstract}
\newpage

\section{Introduction}

The differences in kernels is what makes each one specialized to excel at
performing certain tasks. A kernel might consider video and audio to be higher
priority than a time sensitive process whose quality suffers if the processes are
not executed correctly. This essay will discuss the types of schedulers and
context switching that are responsible for running processes and threads.

\section{FreeBSD Kernel}

\subsection{CPU Scheduling}

The open source operating system FreeBSD supports multiprogramming by using
context switches to determine which thread is executed. The default scheduler,
Timeshare Scheduler, recalculates the priority of the threads based on how much
CPU is to be used and the amount of memory resources it possesses or calls for
in order to be executed.\par
A secondary Real Time Scheduler uses an independent queue that is separated from
the Time Share queue that is not apt to priority degradation, therefore, those
processes are only surpassed by another Real Time whose priority is equal or
higher.\par
The third queue contains processes that run at an idle priority. These threads
are only executed if there are no other Real Time or Time Share threads available
to run.

\subsection{Priority Policy}

The Time Share scheduler utilizes a priority base policy that prefers interactive
programs, such as text editors, over long running threads. Interactive programs
are preferred because of how they preform short bouts of computation quickly
followed by inactivity or I/O. \par
On creation, threads are considered to be high priority at first which allows
them to execute for a fixed time slice. If a thread uses its entire time slice,
it's priority level is lowered, though those that are required to give up their
time slice, like those that perform I/O, are able to keep their higher priority
level. Threads can also be considered low priority if they require a significant
portion of the CPU to execute \cite{3}.

\section{Windows Kernel}

The Windows kernel is similar to the FreeBSD kernel as it uses Real Time Scheduling
as well as thread priority. The difference is that the priority of the threads are
based on a 0 to 31 scale with 0 being the lowest and 31 being the highest. Windows
context switches also play a part in determining which queue of threads are to be
executed by checking if a particular queue is blocked or if there is a queue of
higher priority that has become available to run.\par

\subsection{Scheduling Priority}

There are six classes of process priority levels:

\begin{itemize}
  \item IDLE\_PRIORITY\_CLASS
  \item BELOW\_NORMAL\_PRIORITY\_CLASS
  \item NORMAL\_PRIORITY\_CLASS
  \item ABOVE\_NORMAL\_PRIORITY\_CLASS
  \item HIGH\_PRIORITY\_CLASS
  \item REALTIME\_PRIORITY\_CLASS
\end{itemize}

The default priority of a thread is set to NORMAL\_PRIORITY\_CLASS which is the
third lowest priority class, though the levels can be specified on creation through
the CreateProcess() function. Using this function, it is possible to specify a
process to be IDLE\_PRIORITY\_CLASS for a thread that simply monitors the system
in the background, or HIGH\_PRIORITY\_CLASS for processes that are time critical.
Though making a HIGH\_PRIORITY\_CLASS process should be cautioned as these processes
will take run time away from other processes and prevent them from running.
Likewise, it is generally a bad idea to create a REALTIME\_PRIORITY\_CLASS as
these processes will interrupt the mouse, keyboard, and any disk flushing \cite{1}.

\subsection{Context Switches}

It is possible for a queue of threads to become blocked and are no longer able
to execute. In this event, the scheduler will not allocate any run time to the
threads regardless of how high their priority level is. The scheduler will then
move on the next queue of highest priority in a process called context switching.
The most common reasons for context switches is:
	\begin{itemize}
	  \item A time slice has elapsed
      \item A thread with a higher priority is available to run
      \item A currently running thread has become blocked or is required to wait
	\end{itemize}

\section{Linux Kernel}

The Linux kernel utilizes several different queues for each processor, where the
processor is only capable of running processes from its own queue while the
others have the option of being idle. The queues are also able to balance
themselves so that if one queue is too long, it will move processes to another
queue to reduce its own length. \par
The queue that is given the ability to run is determined by an algorithm that
repeatedly locates the queue of highest priority, calculates its time slice size,
allows that process to run, then sets the process in a list of expired processes
on completing its runtime. A time slice can also be specified when creating a
process using the nice() function. See listing~\ref{lst:label} where the default,
max, and nice() time slices are declared in the Linux header file
\textit{include/linux/sched.h} \cite{2}.

\subsection{The Processor Scheduler}

There are two types of process priorities in the Linux kernel: Static Priority
and Dynamic Priority. Static priorities are used by RT schedulers and are given
a number between 1 and 99 as their priority level. Dynamic priorities are used
by non-RT schedulers whose priority level is calculated by adding the base time
slice and the number of ticks of CPU time the process has remaining before its
time slice expires.

\subsection{Queues and Priorities}

The higher the priority level, the more time that process has to run. For each
priority level, there are two sets of queues allocated for their processes: Active
and Expired. When a process completes its time slice, it is sent to the expired
queue. Once all of the processes in the Active queue has been executed,  the
Active queue switches with the Expired queue.

\begin{lstlisting}[caption={ the default, max, and nice() time slice },label={lst:label}]
#define DEF_COUNTER (10*HZ/100) /* 100 ms time slice */
#define MAX_COUNTER (20*HZ/100)
#define DEF_NICE (0)
\end{lstlisting}


\section{Kernel Comparison}

The three kernels, FreeBSD, Windows, and Linux, all utilize a few similar methods
for handling processes and threads. These common elements are the basic foundation
of keeping a scheduler that can prioritize queues and manage time slices. The
differences between the three kernels are what make the operating system unique and specialized.

\subsection{The Similarities}

Each kernel contains a scheduler that uses a priority system to identify which
processes need to be executed first. This is to insure that processes that need
to be ran immediately are not blocked by a process that is controlling something
that is occurring in the background.\par
Another function they all provide is having at least two types of schedulers
running the processes. One of the most common type of scheduler to have is the
Real Time scheduler that runs the process queues based on priority in a
first-in-first-out method.\par

\subsection{The Differences}

One of the biggest differences between the three kernels is how they assign the
queue their priority levels. While FreeBSD gives higher priority depending upon
the type of process it is, Windows uses a more in depth approach of using an
algorithm to calculate the priority based on clock ticks and time slices. The
Linux kernel also assigns priority based on process type, but it also allows a
user to declare the priority level through the nice() function which creates the
process.\par
Another difference is the number of queues the scheduler is working with. The
Linux kernel has about 140 queues as each priority level receives a active and
inactive queue. The Windows kernel possesses only a couple queues for those that
are going to run, those that are blocked, and those that have finished their time
slice. Similarly, the FreeBSD kernel also uses only a few queues to handle processes.

\subsection{Reason for Differences}

Each of the kernels handles processes a little bit differently which is what
makes the kernel unique and able to specialize in handling certain types of
programs better than other kernels. An example of this is how Windows is
better at running audio and video which allows the user to watch videos and
play video games. FreeBSD is also adept at running audio and video, but is
noticeably slower than that of Windows. Linux, while capable of running a few
video games, specializes in being a light weight operating system that can host
web apps, troubleshoot other computers, and work closely with the hard drive.

\section{Conclusion}

Overall, while Linux, FreeBSD, and Windows operate their kernels with similar
basic concepts and ideas, the differences that lie deeper in their scheduler
organization is what sets them apart. These details allow for the scheduler to
balance user input, I/O, and audio/ video, and seamlessly provide run time for
all the processes waiting in the queue.
\newpage
\bibliographystyle{IEEEtran}

\end{document}
